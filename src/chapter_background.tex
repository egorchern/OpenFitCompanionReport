\chapter{Background}
\label{cha:background}
\section{Literature review}
\subsection{Wearable trackers}
Consumer-based wearable activity trackers are now readily available and can provide individuals with the ability to objectively monitor their physical activity levels. In addition, when combined with the use of smartphone and computer apps, they may assist users through a range of motivational and organisation tools to better manage their personal health \cite{trackersBenefitGeneral}.
There is plenty of research to suggest that using wearable activity trackers helps people become more physically active. A large systematic review has shown that, over several studies, there is a significant increase in daily step count, moderate and vigorous activity as a result of using wearables \cite{trackersBenefitGeneral}. In particular, people with chronic illnesses experienced decreased systolic blood pressure, waist circumference, etc \cite {Franssen2020}; monitoring such patients using wearables after hospitalizations could help detect complications and prevent rehospitalization \cite{hospi}. People of older age can also benefit greatly, showing a moderate increase in physical activity and mobility \cite{SOliveira1188}; there is an indication of good acceptability of such devices among older adults \cite {Franssen2020}. Attractiveness, gamification, readability, and feedback are what drive the health benefits such as commitment to daily physical activity \cite{NELSON2016364}. Wearables are not a magic bullet solution to the fitness problem. Some studies that spanned larger periods of time have observed a decrease in physical activity after an initial positive effect \cite{Finkelstein2016}, or in other words, the use of wearable devices is not directly effective at modifying habitual behaviour \cite{LI2021104487}. Many examined devices did not report sensor accuracy output validity at all, allowing for overestimation or underestimation of metrics \cite{Lee2014ActivityTA}.
\subsection{Nudging}
A nudge is “any aspect of the choice architecture that predictably alters people's behaviour without forbidding any options or significantly changing their economic incentives” \cite{nudgeDef}. An example would be providing information, implementing default choices etc. Nudges are categorised into Type 1 and Type 2. Type 1 nudge typically relies on unconscious thought and a more simple in nature, for example, rearranging the presentations of consumer items in food isles to highlight options that would have ordinarily been ignored \cite{NudgeCritical}. On the other hand, Type 2 nudge relies on conscious thought, those are more complicated and more costly, for example, long-term educational campaigns promoting exercise present the benefits of regular exercise, as well as the harmful effects of continuing to be sedentary. \cite{NudgeCritical}.  In the area of fitness, nudging has been shown to increase physical activity and reduce sedentary behaviour, for example, placing banners that encourage using the stairs displayed positive effects and an increase in stair use \cite{FORBERGER2022106922, Forberger2019}. Nudging has a lower cost of intervention than something like an educational campaign, while still being effective \cite{nudgeCost}. Although nudging has been shown to provide a decently sizable impact, there are doubts about whether it is enough to tackle problems such as fitness on the national (UK) level, raising concern that it may just be a "smokescreen" for inaction at tackling the root causes \cite{Raynerd2177}.

\subsection{Physical Activity}
A highly influential systematic review found "overwhelming evidence (based on millions of participants) that regular physical activity is associated with a reduced risk for all-cause mortality and several chronic medical conditions" \cite{Warburton2017Health}. Another review also supported the previous claim, as well as presenting evidence that physical activity also improves health-related quality of life, functional capacity and mood states \cite{Penedo2005Exercise}. Studies also show that those benefits can be reaped by virtually any person regardless of age, existing physical condition, etc \cite{Penedo2005Exercise, Warburton2017Health}. However, most of the benefits are gained from transitioning from low activity to moderately active \cite{Powell2011Physical}, with some studies pointing that too much physical activity can have negative effects, such as psychological symptoms that mimic depression \cite{Paluska2000Physical} and risks of injury (for untrained individual) \cite{Melzer2004Physical}
The Metabolic Equivalent of Task (MET) is used to express the energy cost i.e. intensity of physical activities as a multiple of the resting metabolic rate \cite{Jetté1990Metabolic}. An individually calculated Resting Metabolic Rate, which is calculated using a person's height, weight, age and gender should be used to calculate a more accurate MET value \cite{Byrne2005Metabolic}; this is the reason that many fitness apps ask for that information. There are tables of typical MET values for different physical activities, however, the manual calculation is inaccurate because you can complete an activity with more intensity than is typically expected \cite{Jetté1990Metabolic}. An MET minute is therefore energy expanded during a minute while at rest, and it is a convenient measure for the amount of physical activity done over a time frame; For example, Walk 2 days a week at 5 METS for 30 minutes per session = 2 x 5 x 30 = 300 MET-minutes \cite{metMinutes}.
\subsection{Sleep}
Several papers confirm the benefits of sleep for mental health, whereby improving sleep quality had a positive effect on reducing mental health difficulties such as depression, anxiety etc \cite{sleep1, sleep2, sleep3}. Disturbances, such as irregular sleep starts can also impact the physical body, weakening the immune system \cite{sleep4} and changing metabolic regulation \cite{sleep2}.

Sleep is made of many stages, but 3 large groups are: light sleep, deep sleep and rapid-eye-movement (REM) sleep; These stages have close ties with heart-rate variability during those stages \cite{sleepDef}. During deep sleep, muscles relax, which promotes their recovery \cite{Jung2010Energy}. REM sleep is essential for normal body physiology, ensuring recovery from sleep and return to consciousness \cite{VERTES1986371}. Wearable devices primarily use heart rate and its variability to determine the sleep stages throughout the night, although other metrics such as skin conductance and temperature may be used as well \cite{Zambotti2019Wearable}. The state-of-the-art method for derivation of sleep stages and sleep quality is using deep learning \cite{Sathyanarayana2016Sleep}.

\section{Similar Systems}
\label{section:similarSystems}
\subsection{Google Fit}
Google Fit is an open ecosystem. It lets developers upload health and wellness data to a central repository where users can access their data from different devices and apps in one location \cite{googleFit}. All types of health data are supported except for something specialised like an ECG. It's defining good qualities are: very clean UI, completely free with no ads and relatively cross-platform app, with iOS and Android supported, but no web version.  Google Fit by itself offers a very limited functionality, such as dashboard of aggregated data - it is a platform. Other apps can connect to that aggregated data and use it for useful purposes, but it may use them unfairly, such as pay-walling or transferring data to US like MyFitnessPal does. Google Fit is still a closed-source system, and it does not make any money, so it is not clear how Google covers those expenses. Google was involved in many legal cases with EU courts for anti-trust \cite{googleAntiTrust} and data protection violations \cite{googleDataProtect, googleDataProtect2} .
\subsection{Apple Health \& Apple Fitness+}
Apple Health is essentially the same as Google Fit, with basic version also just providing a simple dashboard. But, unlike Google, it offers a premium service - Apple Fitness+ with a cost of 9.99£, which offers health reports, trends and exercise plans. It is well integrated in Apple product ecosystem. Biggest downside is that it is only available on iOS devices as an app. Apple Health is a good option for users with iOS device, as the company has a generally good reputation with data privacy, but there were small cases of GDPR violations \cite{CNILApple} and major cases of anti-trust \cite{appleAntiTrust}.
\subsection{MyFitnessPal}
MyFitnessPal is an app described with: "Build healthy habits with the all-in-one food, exercise, and calorie tracker" \cite{fitnesspal}. It's free version contains the same basic functionality, such as syncing data and showing it on a dashboard. Anything more useful like reports, guided exercise routines, etc are behind a premium costing 15.99£ per month. One feature that is very impressive is meal scan, where meals are added by pointing camera at them. It has a negative reputation, experiencing a major data breach involving 150 million users \cite{masuch2021fitness, myFitnessPalDataBreach}, transferring data to the US \cite{myfitnesspalTransferring} where data protection laws are more lenient and pay-walling previously free features \cite{myfitnesspalPaywall}
 
\section{Devices Used}
\subsection{Withings}
\label{section:WithingsWatch}
Withings ScanWatch was used for the project - \cite{withingsStorePage}. It is capable of recording: active minutes (soft, moderate and intense), steps, sleep quality score, sleep durations (light and deep), breathing disturbances during sleep, average heart rate, ECG (electrocardiogram - waveform of heart's electrical activity) and SPO2 (blood oxygenation). ScanWatch integrates a different device called Scan Monitor, which has CE medical certification in Europe and FDA clearance in the United States \cite{withingsStorePage}.  Scan Monitor has the following accuracy metrics published \cite{scanMonitor}:
\begin{itemize}
    \item ECG recording: IEC 60601-2-47 (Requirements for the Basic Safety and Essential Performance of Ambulatory Electrocardiographic Systems.) beat-to-beat QRS detection with F1-score of at least 99.19\% 
    \item AFib: 98.16\% sensitivity in classifying AFib
    \item Sinus Rhythm: 97.20\% specificity in classifying sinus rhythm.
\end{itemize}
SPO2 accuracy is provided by this statement: "The measure of SpO2 in the range 70-100\% has been clinically validated on healthy adult volunteers, at rest against a laboratory co-oximeter.". \cite{scanMonitor}
Other metrics do not seem to have any accuracy declarations. Assuming that those are derived by algorithms, rather than measured directly, by using medically certified measurements. For example, for breathing disturbances the website notes: "Our exclusive algorithm, developed with experts, computes the data from SpO2, heart rate, motion and breathing rate to measure breathing disturbances, an indicator of sleep apnea.". During the app set-up, only when enabling ECG, AFib and SPO2 you had to accept the accuracy document. This leads me to believe that only those metrics are highly accurate and certified medical-grade, while others are not.

The watch has been used in peer-reviewed studies:
\begin{itemize}
    \item Examining if remote monitoring (using the watch + other) improved outcomes of patients undergoing hip and knee arthroplasty: statistically significant reduction in rehospitalization rate in the intervention arm \cite{withingsHospitalization}
    \item Examine large cohorts for presence of Nocturia (waking up to pee at night) in natural settings using the watch: watch made it possible to make association. \cite{withingsNocturia}
    \item Covid confinement adherence monitoring using the watch: used daily steps to find that after 2 weeks compliance decreased. \cite{withingsCovidConfinement}
\end{itemize}


\subsection{Oura}
\label{section:OuraRing}
Oura Generation 2 ring was used for the project - \cite{ouraStorePage}. Oura does not call the device "medical-grade" unlike Withings, since it carries a need for medical certification. Instead, they call it "research-grade". Oura publishes it's sleep tracking accuracy metrics, showing 79\% agreement with gold-standard polysomnography (PSG) for 4-stage sleep classification (wake, light, deep, and rapid eye movement (REM) sleep) \cite{OuraSleepAcc}. Also sleep heart rate and heart rate variability have high degrees of accuracy \cite{ouraHeartAcc}; the study also notes ring's memory limitation, as 2 days old data is lost if user doesn't open the app. Lastly, temperature readings has also been reported to be high accuracy, with near-perfect match with research-grade device at >99\% under lab conditions and 92\% in real world conditions \cite{ouraTemp}; however it is an internal validation study, meaning Oura did it themselves rather than an independent party, and it was done with only 16 individuals over 1 week.

Oura was also used in peer-reviewed studies:
\begin{itemize}
    \item Used Oura ring to measure peripheral temperature of subjects who then reported COVID-19: showed wearable sensors (oura ring) can detect illness in absence of symptoms \cite{smarr2020feasibility}.
    \item Used Oura ring to measure distal body temperature to predict pregnancy earlier than pregnancy test: able to generate hypothetical alert a median of 9 ± 3.9 days prior to the date at which individuals received a positive pregnancy test \cite{ouraPregnancy}.
\end{itemize}
\section{Software Requirements}
There 3 different types of requirements relevant to the domain of software: user requirement describing goal that specific class of user must be able to perform, functional requirement describing what developers must implement to enable users to accomplish tasks (user requirements), Business requirements describing high-level business objective of the organization that builds the system and optionally non-functional requirements, describing more qualitative features the system should have \cite{wiegers2013software}. Business requirements are not relevant to this project, as the main idea is open-source and self-deployment, meaning the project won't bring any commercial value to the creators. 
In general, requirements engineering has shown to improve design performance by facilitating sensemaking (learning about the project context)
User requirements can be understood by non-technical person. They can be represented using user stories, use cases and event-response tables. User story represent requirements using a simple textual template: "As a <role> I can <capability>, so that <receive benefit>" \cite{userStories}. Acceptance criteria may also be attached to a user story, containing details under which testable conditions the story can be considered completed \cite{Kannan2019User}. One of the popular templates for an acceptance criteria is: " Given <conditions>, When <action taken>, Then<outcome of action taken>". Although many practitioners testify that the technique is effective, improving productivity metrics within the team; however, it is only a perceived effectiveness that is not scientific \cite{userStories}.
Functional requirements are more technical, and are intended to be used as description of a system that engineers need to implement. \cite{wiegers2013software}. Although purely textual representations exist, more detailed representations that utilize diagrams are preferred. (System) Use Case Diagram is a UML diagram used to specify functionality offered by the system. \cite{malan2001functional}. Requirements with lots of conditionals can be represented as flow chart, or if the requirement is simple, it could be represented as a sentence in the style of user story. 

\section{Cloud}
Cloud computing is a model for enabling convenient, on-demand network access to a shared pool of configurable computing
resources (e.g., networks, servers, storage, applications, and services) that can be rapidly provisioned and released with minimal management effort or service provider interaction. \cite {cloudDef}
Using cloud for the software service infrastructure has the benefit of off-loading complexity to cloud providers instead of handling it yourself. In comparison with traditional in-house infrastructure, cloud is superior in terms of cost, as there is no upfront cost and it generally being cheaper through economies of scale, as well as and reliability, as maintenance and contingency planning are handled by the best specialists in that area.  \cite{armbrust2009above, hajjat2010cloudward}.  
\subsection{Serverless \& Microservices}
Serverless computing differs from traditional cloud computing, as infrastructure on which services are running are hidden from customers, so that they only need to worry about desired application functionality rather than configuration and management of low-level resources; as well as providing pay-as-you-go model and auto-scaling per demand \cite{serverless1}.  It is mainly used in Event-Driven systems, because successful application of serverless requires well-defined: event, trigger and action \cite{MALAWSKI2020502, serverless2}. 

Monolithic architecture means that an applications runs as a single process in the application server's environment, there maybe multiple copies, but they are just replicas of that one server application. It's main benefit is simplicity, being easier to develop, test and deploy \cite{monolith}.

Microservices architecture on the other hand partitions the functionality of the application, it into a set of small services and making them communicate with each other through light weight mechanisms (e.g., RESTful API or stream-based communications) \cite{fowler2014eb}. The benefits are: scalability, agility, availability and security; for example, when a bug crashes the service in a monolith, the whole service is shut down and the codebase is scanned for the bug, whereas in microservices, only the service with the bug is shut down, and that service may not be fully needed for operations, there could be redirects to another service that is slightly worse, so the service as a whole can still function. However there are also negatives, most notably performance and latency, as communication is usually via network \cite{Li2021Understanding}, therefore slower than monolithic intra-process communication. This approach became popular after Netflix pioneered it and migrated their service to use microservices, after experiencing a catastrophic service outage \cite{monolith}.
% TODO add that it is microservice, adds fault tolerance.
\subsection{Cloud Providers}
There are 3 major providers in this space: AWS, Azure and GCP. They fall into public cloud category, similar to utility services like Electricity, they are available for use to anyone, be it individual developer or a company. We are relying on third-parties to handle things such as legal compliance, disaster management etc. Providers outline their legal promises in the document called Service Level Agreement (SLA) \cite{cloudSLA}.

I chose AWS for this project for the following reasons:
\begin{itemize}
    \item{AWS has the best security system \cite{Narula2015Cloud}, mainly due to IAM service, allowing for tight control of what a resource allowed to access in the infrastructure. As well as being the most trusted provider, never facing major outage or security breach. }
    \item{if used under similar conditions, such as hosting the database in the same region in all comparisons, AWS has among the best quantative performance metrics, such as availability, latency etc \cite{CloudMetrics}  }
    \item {I personally have a lot of experience working with AWS, so I can get started on the project faster.}
\end{itemize}
\subsection{AWS Components}
\begin{itemize}
    \item Lambda: Serverless function offering which allows deployment of microservices without need of managing servers with pay only for what you use pricing structure \cite{LambdaCostSave}, also referred to as Function-as-a-Service model \cite{MALAWSKI2020502}. Allows execution of a function in response to an event, such as direct HTTP call, with management of the underlying resources taken care automatically by cloud provider \cite{MALAWSKI2020502}.  Allows saving up to 77.08\% compared to traditional monolithic server \cite{LambdaCostSave}.
    \item DynamoDB: a fully managed NoSQL database that provides predictable and super-fast performance with unified scalability; Defining Characteristics: Eventually consistent, AP in CAP classification and key-based access focus \cite{DynamoDB}. Tables are schemaless except for the Primary key, which has to be unique among the rows; it may consist of Partition key or composite of Partition key + Sort key, with partition determining internal physical storage partition and items being sorted in order via sort key attribute \cite{awsDynamoWebsite}. 
    \item SNS: Amazon Simple Notification Service (Amazon SNS) is a managed service that provides message delivery from publishers to subscribers (also known as producers and consumers). Publishers communicate asynchronously with subscribers by sending messages to a topic, which is a logical access point and communication channel. Clients can subscribe to the SNS topic and receive published messages using a supported endpoint type \cite{sns}. 
    \item Amplify: AWS Amplify is a complete solution that lets frontend web and mobile developers easily build, connect, and host fullstack applications on AWS \cite{amplify}. Although it has a lot of services, such as no-code development solution, only Amplify Hosting will be used. It provides git-based workflow for continious deployment and subsequent hosting: performing typical front-end server functionalities: file compression, caching, responding with page's files. 
\end{itemize}
The following diagram gives visual representation of each above-mentioned component. \ref{fig:awsComponents}
\begin{figure}
    
    \centering
    \includegraphics[width=0.95\textwidth,keepaspectratio]{../images/AWS_components.pdf}
    \caption{AWS diagram component labelling}
    \label{fig:awsComponents}
\end{figure}
\subsection{Security}
Software Security is important, especially in the context of this project that deals with sensitive health data. Generic web-accessible recommendations should apply to this project, such as encryption, xss, etc. However, deploying the application on the cloud presents new security related challenges, which a lot of companies suffer from [cite]. 
\section{LLMs}
"A large language model is the language model with massive
parameters that undergoes pretraining tasks (e.g., masked
language modeling and autoregressive prediction) to un-
derstand and process human language, by modeling the
contextualized text semantics and probabilities from large
amounts of text data " \cite{Yao2023ASO}
Unlike neural networks, the main way to improve response quality is to improve the prompt \cite{Liu2021PretrainPA}; "Prompt Engineering" is the technique for doing that, maximizing utility of LLMs in various tasks \cite{zhou2023large}. The following techniques were discovered:
\begin{itemize}
   \item Retrieval Augmented Generation (RAG): RAG is a functionality that allows LLMs to access relevant external knowledge and use it for response generation. It is usually more prioritised that the ordinary context,significantly reducing the chances of hallucinations and improving output quality \cite{gao2024retrievalaugmented}.
   \item Few Shot prompting: Also called in-context learning, providing hand-crafted examples of good replies to an input, has shown to provide better model performance; however it still struggles with complex reasoning tasks \cite{brown2020language, min2022rethinking}. When 1 pair is provided it is called 1-shot prompt, 5 - 5-shot, etc. 
   \item Chain of Thought (CoT) prompting: when utilising few shot prompting, providing intermediate reasoning steps in the example reply has shown to improve performance as well as enabling complex reasoning capabilities \cite{wei2023chainofthought}. It is also possible in a zero-shot prompt i.e prompt without examples, by providing this "magic" phrase at the end of the prompt: "Let's think step by step"; This has shown to improve performance over default prompt and rival few-shot prompts \cite{kojima2023large}.
   \item Details, Persona: According to OpenAI \cite{openAI}, including more details and asking the model to assume a persona, like a personal trainer, can improve reply relevancy. 
\end{itemize}
 In 2022, OpenAI made GPT3 available to the public, a product that surpassed Google in daily webpage visits. It revolutionized AI, in a way that a lay person could use it with some decent efficiency. The vital factor is it's ability to effectively utilize context window, a collection of prior information (such as prior messages sent in a chat) that is used to influence the next output. GPT4 is an improvement though with a higher price. The workflow of using GPT4 as a software developer is as follows: Explicitly attach context messages together with a prompt. The complexity of managing contexts was upon your system that uses the OpenAI API. This recently changed with Assistants API. This feature allows creating of assistants, which can have threads that correspond to continuous chat with a user. Key factor is that the complexity of maintaining context is handled by OpenAI. Also it features OpenAI's own Retrieval Assisted Generation (RAG) solution named Knowledge Retrieval. 

 One limitation of Large Language Model (LLM) content generation is hallucination, or false assertions in the generated text \cite{ji2023survey}.

\section{Miscellaneous}
\label{section:goldenRules}
Ben Shneiderman's eight golden rules of UI design are widely regarded as good framework for designing and evaluating UIs. It comprises of 8 simple rules that a good UI should follow \cite{goldenRulesUI}: \begin{itemize}
    \item Strive for consistency: page contents should mostly be consistent, such as fonts, colors etc. 
    \item Seek universal usability: everyone can use the product
    \item Offer informative feedback: user's actions should have observable impact.
    \item Design dialogs to yield closure: give confirmation that something failed or succeeded, never leave users not knowing what is going on.
    \item Prevent errors: when possible, restrict actions available so that user's can't make obvious errors.
    \item Permit easy reversal of actions: actions should be reversible as much as possible
    \item Keep users in control: make it easy for users to accomplish their desired result.
    \item Reduce short-term memory load: don't show too much information at once.
\end{itemize}

\section{Data analysis \& Statistical Testing}
\subsection{Bland-Altman Plot}
Bland-Altman plot is a general method for visualizing differences between measurements of two methods. The Bland-Altman (BA) graph consists of a scatter plot in which the difference between two measures (Test \#1 - Test \#2) is constructed on the vertical axis, while the mean of the two measures ([Test \#1 + Test \#2]/2) is depicted on the horizontal axis \cite{kaur2017bland}. Limits of agreement may also be drawn. They signify boundaries of difference, beyond which points would be more than x standard deviations away from the mean difference; 1.96 SD is commonly used \cite{myles2007using}.  In particular, it is used extensively in medical research, when there is a need to determine if two methods can be used interchangeably \cite{myles2007using}.
\subsection{Outlier detection - Z-score}
Z-score or also known as Standard score, is the number of standard deviations by which a data point is above or below the mean of the population. Ideally, it requires knowing population mean and standard deviation (SD), however, for practical reasons those are estimated by sample's mean SD; It is calculated by: $z=\frac{x-\hat{x}}{SD}$ \cite{zscoreBook}. It can be used for outlier detection, such as data points more than 2 SDs away are considered outliers, i.e data points with $\text{abs}(z) >= 2$ are outliers.
\subsection{Hypothesis testing}
Student's t-test is a statistical testing method to test a hypothesis of whether population means of 2 groups are different with a certain confidence. It is often the preffered choice over Z-test, as it does not require knowing population's mean and SD; instead those are estimated from the sample and corrected with degrees of freedom according to the sample size. The normal version, also called Independent t-test assumes equal variance between 2 groups, whereas Paired version does not assume equal variance. Paired version should be used when two groups depend on one another. \cite{LIVINGSTON200458}. 

Two one-sided test (TOST) also allows testing whether population means of 2 groups are different, however we can also define bounds on what is considered a worthwhile difference - these are called equivalence bounds; if the difference between 2 groups is within those defined lower and upper equivalence bounds at a certain confidence, the difference is not too interesting. Basically we can test whether the difference between groups is significantly more than some defined tolerance values, instead of just testing whether difference is more. \cite{tost}.
% On 4th March 2024, Shortly after finishing the project implementation, Anthropic announced Claude 3 Sonnet. It's benchmark scores are similar or better than gpt-4-1106-preview, it has bigger context window but costs 333\% less. This makes the reasoning behind the choice of gpt-4-1106-preview not valid at the moment of writing the report, however it was valid at the time of planning for AI functionality at the time of Dec 2023. 
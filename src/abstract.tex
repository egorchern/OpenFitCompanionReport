%\abstracttitle
% Single spacing can be turned on for the abstract
%
{\singlespacing
Smart health-tracking devices and apps that leverage them have gained popularity as people become more health-conscious. However, due to the restrictiveness of many companies, the data is dispersed and not utilised effectively for guidance towards a healthier lifestyle. Many commercial apps try to tackle this problem, but most are inadequate in some criteria, such as transparency or cost. Profit incentives prevent users from receiving the best experience. The report presents OpenFitCompanion, an open-source, self-deployable, cross-platform application, designed as an event-driven composition of serverless microservices. It integrates data from health trackers to create helpful and highly personalised insights using a mixture of algorithms and Large Language Models (LLM). Evaluating the final artefact for various criteria such as cost, security and more, has shown that the application is a strong competitor to the free apps on the market, but fails to present a better utility-to-cost ratio against the paid apps. Nevertheless, it provides a strong foundation on which further improvements and optimisations can be applied. 

Enabled by the core application, a research question: ”Are examined devices different?” was investigated using two months of personal fitness data. Framing the problem as an equivalence test and performing t-tests displayed evidence for significant difference in the measurement of soft, moderate and intense activity seconds. Also showing that devices have significantly different sleep stage classification frameworks. However, a more rigorous study would be required to draw any conclusions, as user errors and low sample size could have had an impact on the results. 

We hope that OpenFitCompanion advances the domain of health-tracking apps towards more open architectures and facilitates further studies involving health-tracking devices. 
}


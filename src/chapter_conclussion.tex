\chapter{Conclusions}
\label{cha:evaluation}
\section{Summary of Achievements}
In summary, all project aims were fulfilled, except for the cost of service and ease of set-up aims. Other commercially available products cost less, while providing similar or better functions, and the service is very tough to self-deploy.

Core application - OpenFitCompanion was produced, an open-source alternative to health data aggregators and health companions. Providing convenient access to all data in one place with an option to export it to a file. Employing nudging mechanisms via daily and weekly reports, where users can effortlessly examine: trends and progress towards their goals. Leveraging AI to create activity plans, curating exercises to align with user's preferences and needs, with exercise reminders sent via push-notifications. Lastly, enabling a natural language conversation about user's health data, mimicking interaction with a personal trainer at a fraction of the cost. Architecture was explained with the help of diagrams and key decisions justified. 

Data analysis was performed in order to answer research question concerning the equivalence of used devices. Finding of exploration were presented: Bland-Altman plots, implications of initial findings and outlier protocol. Empirical difference percentages were calculated. Hypothesis testing problem was formulated, with the justification of statistical test type used. Finally, results were interpreted, and possible conflicts that could have caused the results presented. 
\section{Reflection}
\section{Further Work}
\subsection{Reducing AI inference cost}
Mixed evaluations (BIG-Bench-Hard) 82.9\% vs 83.1\% of gpt4.
\subsection{Diet via LLM with vision}
better at vision as well. ANLS score 89.5\% vs 88.4\% of gpt4.
\subsection{Avoiding Cloud vendor lock-in \& Automating Deployment}
OpenStack more interoperable open metal, Terraform - less but is easier to work with.
\subsection{Testing}
